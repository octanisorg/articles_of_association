%!TEX program = xelatex

\documentclass[12pt,a4paper,oneside]{article}
\usepackage{paracol}
\usepackage{fontspec}
\usepackage{microtype}
\usepackage{graphicx}
\usepackage{enumerate}
\graphicspath{ {images/} }
\usepackage[margin=2cm]{geometry}
%\usepackage{navigator}
\usepackage[utf8]{inputenc}
\usepackage{color}

\newcounter{art}
\addtocounter{art}{1}

\newcommand{\english}{    \switchcolumn[0]\noindent}
\newcommand{\french}{    \switchcolumn[1]\noindent}
\renewcommand{\thesubsubsection}{Art. \arabic{art} }
\renewcommand{\thesubsection}{\arabic{subsection}}


\setmainfont{Gibson Light}
\newfontfamily\semibold{Gibson}

\newcounter{para}
\newcommand\mypara{\par\refstepcounter{para}\textsubscript{\thepara}\space}



\setcounter{section}{1}
\begin{document}

\begin{center}
	\includegraphics{octanis_org_logo_large}
\end{center}

\begin{paracol}{2}\sloppy

\french
	\section*{Statuts de l’Association}


\french
	\subsection{Nom, Siège et Buts}
	\subsubsection{Nom}\stepcounter{art}
	L’association \textbf{Octanis} (ci-après « l’association »), est une association à but idéal constituée conformément aux dispositions des articles 60 et suivants du Code civil suisse.


\french
	\subsubsection{Siège}\stepcounter{art}
	Le siège de l’association est à Lausanne.


\french
	\subsubsection{Buts}\stepcounter{art}
	$^1$ L’association a pour buts:
	\begin{enumerate}[(a)]
		\item Promouvoir le concept de «rapid prototyping» et de «design thinking» en développant des missions, produits et prototypes par ces approches ou des similaires.
		\item Donner à ses membres l’accès aux ateliers de l’association, aux pièces, matériel, outils, à l’imprimante 3D, à la station de soudage et autres pour créer leurs prototypes. 
		\item Aider ses membres à développer un esprit entrepreneurial.
		\item Redistribuer l’expérience acquise au travers d’ateliers.
		\item Contribuer à la communauté «open source» en mettant à disposition des logiciels et de la documentation lorsque c’est possible.
		\item Fournir des activités sur les campus de l’EPFL ainsi que pour d’autres écoles.
	\end{enumerate} 
	$^2$ L’association est affranchie de toute orientation et rattachement de nature politique ou religieuse.

\french 
	\subsection{Membres}


\french
%	\subsubsection{}
	\stepcounter{art}
%	Chaque membre a l’obligation de connaître et de comprendre le Code de Conduite de l'association.



\french
	\subsubsection{}\stepcounter{art}
Les étudiants de l’EPFL, de l’UNIL ou et des autres Hautes Ecoles suisses  peuvent être admis comme membre de l’associationD’autres personnes peuvent être admises pour autant que le sociétariat reste composé de 50\% d’étudiants de UNIL et EPFL.



\french
	\subsubsection{}\stepcounter{art}
	$^1$ L’admission d’un nouveau membre est de la compétence du comité, avec possibilité de recours à l’assemblée générale en cas de refus.  \\
	$^2$ La demande d’admission est présentée par écrit au comité. \\
	$^2$ Par sa demande d’admission, le candidat adhère sans réserve aux statuts et au code de conduite de l’association et s’engage à respecter les décisions de l’assemblée générale et du comité.


\stepcounter{art}
\french
	\subsubsection{}\stepcounter{art}

	$^1$ La qualité de membre se perd par démission ou par exclusion sur décision de l’assemblée générale. \\
	$^2$ Le membre peut démissionner en tout temps de l’association. L’annonce de la démission est présentée par courriel au comité. \\
	$^3$ Sur proposition du comité, l’assemblée générale peut exclure un membre qui contrevient gravement aux buts ou aux intérêts de l’association.

\newpage 

\french 
	\subsection{Ressources}


\french
	\subsubsection{}\stepcounter{art}
	Les ressources de l’association sont constituées par les cotisations des membres, les recettes des manifestations organisées par l’association, par les subventions, les parrainages, les dons ou les legs, les recettes tirées des projets que l’association effectue pour des tiers ainsi que par toute autre recette.

\stepcounter{art}

\french
	\subsubsection{}\stepcounter{art} 
	Les cotisations sont comme décrites ci-dessous :
	\begin{enumerate}[(a)]
	\item Membre hors comité: 50.- annuellement
	\item  Membres du Comité: 100.- annuellement
	\end{enumerate}


\french 
	\subsection{Comptabilité et bilan}
	\subsubsection{}\stepcounter{art}

	$^1$ L’association tient une comptabilité et un bilan. \\
	$^2$ Le trésorier présente à l’assemblée générale la comptabilité et le bilan annuel avec le rapport des vérificateurs aux comptes.  


\french
	\subsection{Organisation}

\french
	\subsubsection{}\stepcounter{art}
	Les organes de l’association sont l’assemblée générale (ci-après « AG »), le comité et les vérificateurs aux comptes.

\french
	\subsubsection{L’assemblée générale}\stepcounter{art}
	$^1$ L’AG réunit les membres de l’association. \\
	$^2$ L’AG est le pouvoir suprême de l’association. Elle a pour tâches et compétences, celles qui ne sont pas attribuées à un autre organe, soit notamment :
	
	\begin{enumerate}[-]
	\item élire les membres du comité et les vérificateurs aux comptes ;
	\item se prononcer sur l’admission des nouveaux membres sur recours et sur l’exclusion des
    membres ; 
	\item décider des activités de l’association en rapport avec ses buts ;
	\item fixer le montant des cotisations ;
	\item approuver le budget, la comptabilité et le bilan annuel, ainsi que le rapport du comité
    de direction ;
	\item déterminer le montant maximum à hauteur duquel le comité peut engager
    l’association ;
	\item disposer des actifs sociaux ;
 	\item modifier les statuts ;
 	\item prononcer la dissolution de l’association ;
 	\item toute autre affaire à régler.
	\end{enumerate}

\french
	\subsubsection{}\stepcounter{art}

	$^1$ L’AG se réunit en séance ordinaire au moins une fois par an, dans les trois mois qui suivent la clôture du dernier exercice comptable. Elle est convoquée par le comité, par avis donné trois semaines à l’avance. \\
	$^2$ Une AG extraordinaire est convoquée à chaque fois que le comité l’estime opportun ou à la demande des vérificateurs aux comptes ou d’un cinquième des membres de l’association. \\
	$^3$ La convocation à l’AG mentionne sa date, son lieu et son ordre du jour. \\
	$^4$ Sauf disposition contraire des statuts, l’AG siège valablement quel que soit le nombre des membres présents. \\
	$^5$ L’AG est présidée par le président de l’association ou, s’il y a lieu, par le vice-président ou un autre membre du comité. \\
	$^6$ Les décisions de l’AG sont consignées dans son procès-verbal. \\
	$^7$ L’AG peut prendre des décisions par voie de circulation, notamment au moyen d’une plateforme informatique de vote en ligne.

\french
	\subsubsection{}\stepcounter{art}
	$^1$ Chaque membre dispose d’une voix à l’AG. \\
 	$^2$ L’AG décide à la majorité simple des voix exprimées. En cas d’égalité des voix, la voix du président est prépondérante. \\
	$^3$ L’AG élit les membres du comité à la majorité absolue des voix exprimées au premier tour et à la majorité relative au second tour. \\
	$^4$ L’AG décide de l’admission sur recours et de l’exclusion de membres à la majorité absolue des voix exprimées. \\
	$^5$ L’AG modifie les statuts à la majorité des deux tiers des voix exprimées. \\
	$^6$ L’AG prononce la dissolution de l’association à la majorité des deux tiers des voix exprimées lors d’une AG extraordinaire convoquée spécialement à cet effet et réunissant au moins la moitié des membres. Si ce quorum n’est pas atteint, une AG extraordinaire est convoquée à nouveau dans un délai de vingt jours. Elle siège alors quel que soit le nombre des membres présents.


\french
	\subsubsection{Le comité}\stepcounter{art}
	$^1$ Le comité est l’organe exécutif de l’association. Il se compose de trois à sept membres, dont le président, le vice-président et le trésorier. \\
	$^2$ Les membres du comité sont élus par l’AG parmi les membres de l’association, pour une durée d’un an renouvelable. Au moins l’un d’entre eux doit, si possible, avoir été membre du comité lors du dernier mandat.
 

\french
	\subsubsection{}\stepcounter{art}
	Le comité a les tâches suivantes:
	\begin{enumerate}[-]
		\item administrer l’association;
		\item exécuter les décisions de l’AG;
		\item diriger, coordonner et représenter l’association;
		\item gérer les ressources et le budget;
		\item tenir la caisse;
		\item tenir la comptabilité et le bilan;
		\item veiller au bon fonctionnement de l’association;
		\item sauvegarder les intérêts de l’association;
		\item rapporter son activité à l’assemblée générale.
	\end{enumerate}

\stepcounter{art}

\french
	\subsubsection{}\stepcounter{art}
	Le comité engage l’association par la signature collective à deux du président ou du vice-président et d’un second membre du comité. 

\french
	\subsubsection{}\stepcounter{art}
	$^1$ Le comité se réunit sur convocation du président aussi souvent que la conduite des affaires l’exige. Il doit être convoqué si deux membres du comité au moins le demandent. \\
	$^2$ Le comité ne peut délibérer qu’à la condition que le président ou le vice-président soit présent.\\
	$^3$ Les décisions du comité sont consignées dans son procès-verbal.\\
	$^4$ Le comité prend ses décisions à la majorité absolue des membres présents. En cas d’égalité des voix, la voix du président est prépondérante.\\
	$^5$ En cas d’urgence, le comité peut prendre des décisions par voie de circulation, pour autant qu’aucun de ses membres ne s’y oppose.


\french
	\subsubsection{Les vérificateurs aux comptes}\stepcounter{art}
	$^1$ Deux vérificateurs aux comptes sont élus par l’AG parmi les membres de l’association, pour une durée d’un an renouvelable.\\
	$^2$ Les vérificateurs sont chargés de soumettre à l’AG un rapport sur les comptes qui lui sont présentés. Ils peuvent en tout temps vérifier l’état de la caisse, obtenir la production des livres et pièces comptables, ainsi que convoquer une AG extraordinaire


\french
	\subsection{Dissolution}

	\subsubsection{}\stepcounter{art}
	$^1$ En cas de dissolution de l’association, le mandat de liquidation revient au comité en fonction. \\
	$^2$ L'actif net disponible est entièrement versé à une association d’étudiants ayant des buts similaire à ceux de l’association, choisie en accord avec l’EPFL.


\french
	\subsection{Dispositions finales}

	\subsubsection{}\stepcounter{art}
	Les présents statuts sont édictés en français et publiés sur le site internet de l’association. Les présents statuts ont été adoptés par l’assemblée générale du 23 Novembre 2016.


 	\vspace{\fill}

	\noindent
	Lausanne, le 23 Novembre 2016.
	\vspace {1.5cm}

	\noindent
	\hrulefill \\
 	Le président \\

	\vspace {1.0cm}
 	\noindent 
 	\hrulefill \\
 	Le vice-président\\


\clearpage
% ================================
% EMBED CODE OF CONDUCT

% \input{../code_of_conduct/code_of_conduct_content_english_french}



\end{paracol}



\end{document}

